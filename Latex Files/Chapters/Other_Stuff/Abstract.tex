\documentclass[../../main.tex]{subfiles}
\begin{document}
%
    The cytoskeleton is an active soft material, comprised and regulated by a vast number of proteins. Conferring structural stability to the cell, it also plays a relevant role in a variety of other cell processes. Its diverse behaviour has made the cytoskeleton an interesting case study in soft matter physics. 
    
    In this work, we aim at explaining the distinct rheological properties through the binding and unbinding of cross-linkers. We start with the Gaussian chain model, where we compare results from our numerical simulations with those found in literature. To model the cross-link capable monomers, we first study particle aggregation, implementing a new interaction potential and investigating the role of temperature and cross-linker density in the size of aggregates. By combining the two studies, we start working towards a reticulated polymer network, first studying placement of cross-linkers in the chain and then adding steric repulsion to chain interactions.
    
    This work is a simplified approach to the study of a cross-linked polymer. Despite simple, this model sets the ground for future studies, where more system-specific interactions should be included. 
    \newline
    \newline
    \textbf{Keywords:} cross-linker; cytoskeleton; polymer.
%
\end{document}
\documentclass[../../main.tex]{subfiles}
\begin{document}
\begin{otherlanguage}
{portuguese}
%
    O citoesqueleto é um biopolímero, composto e regulado por um vasto número de proteínas. Responsável por uma variedade de processos celulares, entre os quais o movimento, divisão e crescimento celular, é no entanto na estabilidade mecânica que confere à célula que tem a sua principal função.

    Avanços em várias técnicas experimentais vieram mostrar um comportamento reológico complexo e diversificado, muitas vezes contrastante com os seus homólogos sintéticos. A procura por um modelo que explique as peculiaridades deste biopolímero tornou-se, nos últimos anos, um dos temas mais ativos na física da matéria condensada mole.

    Na base do modelo desenvolvido ao longo deste trabalho está a importância da anexação e separação das diversas cadeias poliméricas na modelação de uma resposta a estímulos externos. Esta dinâmica é promovida por uma variedade de proteínas acessórias as quais designamos de reticulações. 

    Neste trabalho começámos por introduzir o modelo da cadeia Gaussiana, mostrando a sua equivalência com o modelo mecanicista correspondente ao uso de monómeros pontuais ligados por molas. Provou-se que o modelo computacional implementado reproduzia o comportamento de uma cadeia polimérica que foi previsto, não só neste trabalho, como na literatura. 

    Em paralelo, estudámos a agregação de partículas, implementando um potencial de interação que visa simular as características de uma reticulação. Após a validação do potencial, investigámos o efeito da temperatura e densidade de reticulações no número de ligações formadas. Mostrámos a existência de intervalos de valores que maximizam esta quantidade para os parâmetros do potencial escolhidos.

    Ao combinar os dois estudos feitos foi-nos finalmente possível iniciar a investigação ao modelo reticulado. Após comparar diferentes posicionamentos para a reticulação na cadeia, foram adicionados efeitos estéricos na forma de repulsão entre diferentes cadeias. 

    Este trabalho corresponde à primeira abordagem de um modelo computacional para um polímero reticulado. Daqui resulta um modelo simples mas fundamentado, o qual deve ser trabalhado no futuro, com a adição de resistência à flexão das cadeias, um estudo mais pormenorizado no uso de repulsão entre cadeias e a adição de uma tensão de cisalhamento.
    \newline
    \newline
    \textbf{Palavras-chave:} citoesqueleto; polímero; reticulação.
%
\end{otherlanguage}
\end{document}
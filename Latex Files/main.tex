\documentclass[11pt,a4paper,twoside,openright]{book}

%%%% Usar vários subfiles %%%%
\usepackage{subfiles}

%%%% Usada para comentar um bloco de texto %%%%
\usepackage{comment}

\usepackage[portuguese,english]{babel}
\usepackage[utf8]{inputenc}

%%%% Fonte mais próxima de Times New Roman %%%%
\usepackage{mathptmx}

%%%% Alterar margens do documento %%%%
\usepackage[margin=2.5cm]{geometry}
\renewcommand{\baselinestretch}{1.15}
\usepackage{setspace}
\usepackage{fancyhdr}
\pagestyle{fancy}
\renewcommand{\chaptermark}[1]{\markboth{\MakeUppercase{\thechapter. #1 }}{}}
\renewcommand{\sectionmark}[1]{\markright{\thesection\ #1}}
\fancyhf{}
\fancyhead[RO]{\bfseries\rightmark}
\fancyhead[LE]{\bfseries\leftmark}
\setlength{\headheight}{15pt}
\fancyfoot[C]{\thepage}
\renewcommand{\headrulewidth}{0pt}
\renewcommand{\footrulewidth}{0pt}
\addtolength{\headheight}{0.5pt}
\fancypagestyle{plain}{
  \fancyhead{}
  \renewcommand{\headrulewidth}{0pt}
}


%%%% Package que permite incluir imagens %%%%
\usepackage{graphicx}
\graphicspath{{Figures/}{../Figures/}} % path para a pasta onde estão as imagens
\usepackage[nottoc,numbib]{tocbibind}
%\usepackage{subfig} % support for the manipulation and reference of small or ‘sub’ figures and tables within a single figure or table environment
\usepackage{float} % Now we can put an [H] in \begin{figure} to specify the exact location of the figure

%%%% Para fazer referências %%%%
\usepackage[backend = biber, style = nature, sorting = none, citestyle = numeric-comp]{biblatex} % basic style, author-year citations
%\addbibresource{tese.bib} %Imports bibliography file
\addbibresource{Revisao_Bibliografica.bib} %Imports bibliography file
\usepackage{csquotes} % ensure that quoted texts are typeset according to the rules of main language
\usepackage{hyperref} % produce hypertext links in the document

%%%% Criar uma pag com as constantes usadas %%%%
\usepackage[intoc]{nomencl}
\makenomenclature
\renewcommand{\nompreamble}{\markboth{NOMENCLATURE}{Nomenclature}}

\usepackage{etoolbox}
\renewcommand\nomgroup[1]{%
  \item[\bfseries
  \ifstrequal{#1}{L}{Notation}{%
  \ifstrequal{#1}{A}{Abbreviations}{}}]}%


%%%% Indentar primeiro parágrafo %%%%
\usepackage{indentfirst}

%%%% Permite remover os espaços entre itens %%%%
\usepackage{enumitem}

%%%% Caption para imagens %%%%
\usepackage[font=footnotesize]{caption}

%%%% Referenciar anexos %%%%
\usepackage[toc,page]{appendix}

%%%% Adicionar pdfs %%%%
\usepackage{pdfpages}

%%%% Alterar nome de capitulo para a referencia %%%%
\usepackage{fmtcount}

%%%% Used to write code %%%%
\usepackage{minted}
\usemintedstyle{emacs}

\usepackage{url}

%%%% Math stuff %%%%
\usepackage{mathtools}
\newcommand\ddfrac[2]{\frac{\displaystyle #1}{\displaystyle #2}} % forces TeX to render the material in display-style math mode for the numerator and denominator
\newcommand\mean[1]{\mathop{\overline{#1}}} % gives me a bar over the letter with a \mean command
\newcommand{\norm}[1]{\left\lVert#1\right\rVert} % gives the norm of a vector
\usepackage{mathrsfs} % fancy math letters
\usepackage{bigints} % big integral signs
\usepackage{amssymb}
\usepackage{amsmath}
\usepackage{bm} % bold
%\usepackage{systeme} % makes systems of equations or inequalities
\DeclareMathOperator{\erf}{erf} % error function operator
% unit vector shit
\newcommand{\uveci}{{\bm{\hat{\textnormal{\bfseries\i}}}}}
\newcommand{\uvecj}{{\bm{\hat{\textnormal{\bfseries\j}}}}}
\DeclareRobustCommand{\uvec}[1]{{ % unit vector
  \ifcat\relax\noexpand#1%
    % it should be a Greek letter
    \bm{\hat{#1}}%
  \else
    \ifcsname uvec#1\endcsname
      \csname uvec#1\endcsname
    \else
      \bm{\hat{\mathbf{#1}}}%
     \fi
   \fi
}}

\usepackage{xcolor}
\usepackage{gensymb} % Provides generic commands (e.g. \degree, \celsius, \perthousand, \micro and \ohm)

\usepackage{filecontents}
\usepackage{float}

\raggedbottom

\usepackage{subcaption}
\usepackage{eurosym}

\usepackage{booktabs}

\usepackage{cleveref} % format of cross-references to be determined automatically according to the “type” of cross-reference (equation, section, etc.)
\crefname{appsec}{appendix}{Appendices} % \cref uses appendix when referring, guess what?, the appendices 

\makeatletter
\def\cleardoublepage{\clearpage\if@twoside \ifodd\c@page\else
    \hbox{}
    \thispagestyle{plain}
    \newpage
    \if@twocolumn\hbox{}\newpage\fi\fi\fi}
\makeatother \clearpage{\pagestyle{plain}\cleardoublepage}

\begin{document}
%\onehalfspacing
\begin{sloppy}
%-----------------------------------------------------------------
%
% Capa (FRENTE)
%
%------------------------------------------------------------------
\begin{titlepage}
\topmargin = -55pt
\centering
\Large UNIVERSIDADE DE LISBOA\\
\Large FACULDADE DE CIÊNCIAS\\
\Large DEPARTAMENTO DE FÍSICA\\
\vfill
\begin{figure}[ht!]
\centering
\includegraphics[scale=0.5]{ciencias_ulisboa_marca_horizontal_rgb_preto.png}
\end{figure}
%
\vfill
\textbf{\LARGE Dynamics of Active Polymer Networks}
\vfill
\Large Francisco Cerdeira
\vfill
\textbf{\Large Mestrado Integrado em Engenharia Biomédica e Biofísica}\\
Perfil Biofísica Médica e Fisiologia de Sistemas\\
\vspace{0.8cm}
%\Large Versão Provisória\\
\vspace{0.8cm}
\Large Dissertação orientada por:\\
\Large Prof. Doutor Nuno Araújo\\
\Large Prof. Doutor Pedro Patrício\\
\vfill
\Large \the\year\\
\end{titlepage}


%-----------------------------------------------------------------
%
% Capa (TRÁS)
%
%------------------------------------------------------------------
\clearpage \thispagestyle{empty}\mbox{}\clearpage
\pagenumbering{roman}
%\newpage
%\thispagestyle{plain}
%\mbox{}


%-----------------------------------------------------------------
%
% AGRADECIMENTOS
%
%------------------------------------------------------------------
\newpage
\thispagestyle{plain}
\chapter*{Acknowledgements}
    \subfile{Chapters/Other_Stuff/Acknowledgements}

%-----------------------------------------------------------------
%
% ABSTRACT
%
%------------------------------------------------------------------
\newpage
\thispagestyle{plain}
\chapter*{Abstract}
    \subfile{Chapters/Other_Stuff/Abstract}


%-----------------------------------------------------------------
%
% RESUMO
%
%------------------------------------------------------------------
\newpage
\thispagestyle{plain}
\chapter*{Resumo}
    \subfile{Chapters/Other_Stuff/Resumo}
%
%-----------------------------------------------------------------
%
% ÍNDICE
%
%------------------------------------------------------------------
%
% TABLE OF CONTENTS
\newpage
\thispagestyle{plain}
\renewcommand{\contentsname}{Contents}
\tableofcontents
% LIST OF FIGURES
\newpage
\thispagestyle{plain}
\listoffigures
% LIST OF TABLES
\newpage
\thispagestyle{plain}
\listoftables
% NOMENCLATURE
\newpage
\thispagestyle{plain}
\subfile{Chapters/Other_Stuff/nomecl}
%\clearpage 
\thispagestyle{plain}\mbox{}\clearpage
%
\pagenumbering{arabic} %numeros em numeração árabe
%
%-----------------------------------------------------------------
%
% INTRODUÇÃO
%
%------------------------------------------------------------------
%
\newpage
\thispagestyle{plain}
%
\chapter{Introduction}\label{ch:intro}
    \subfile{Chapters/Other_Stuff/Intro}
%
%-----------------------------------------------------------------
%
% BROWNIAN DYNAMICS
%
%------------------------------------------------------------------    
%
\chapter{Collective Dynamics of Particles in Solution}\label{ch: polDynamics}
    \subfile{Chapters/Theory/Brownian_Motion}
    \subfile{Chapters/Theory/Gaussian_Chain}
\chapter{Particle-Particle Aggregation}\label{ch: Particle-Particle Aggregation}
    \subfile{Chapters/Theory/LJ_2}
%
%-----------------------------------------------------------------
%
% CROSSLINKED POLYMER NETWORKS
%
%-----------------------------------------------------------------
%
\chapter{Crosslinked Polymer Networks}\label{ch: CrosslinkNet}
    \subfile{Chapters/Results/Crosslinked_Network}
%
%-----------------------------------------------------------------
%
% CONCLUSIONS
%
%-----------------------------------------------------------------
%
\chapter{Conclusions \label{ch:conclusion}}
    \subfile{Chapters/Other_Stuff/Conclusions}
    
%
%-----------------------------------------------------------------
%
% REFS
% 
%-----------------------------------------------------------------
%
%\bibliographystyle{apacite}      % basic style, author-year citations
%\bibliography{tese.bib}
\printbibliography[heading=bibintoc]

%    
%-----------------------------------------------------------------
%
% APPENDIX
%
%------------------------------------------------------------------      
%
\subfile{Chapters/Other_Stuff/Appendix}

\end{sloppy}
\end{document}
